\documentclass{article}

\usepackage{amsfonts,amsmath,amssymb}
\usepackage{hyperref}

\author{Ali Dorostkar \and Anastasia Kruchinina}
\title{\textsc{A new teaching style in computer programming}}
\date{\today}

\begin{document}

\maketitle \begin{abstract}
In this project we introduce a teaching technique for fields that require
practical work in order to be understood deeply. The focus of this article is 
to provide ways to simulate deep learning among students. To this end we describe 
details of a lecture session tailored specifically for Programming courses
where the students learn by participating, participating and other methods.
\end{abstract}

\section{Introduction} % (fold)
\label{sec:introduction}

``I have always wished for my computer to be as easy to use as my telephone;
my wish has come true because I can no longer figure out how to use my
telephone. - Bjarne Stroustrup''
\newline  
\newline
``For a long time it puzzled me how something so expensive, so leading edge,
could be so useless. And then it occurred to me that a computer is a stupid
machine with the ability to do incredibly smart things, while computer
programmers are smart people with the ability to do incredibly stupid things.
They are, in short, a perfect match. - Bill Bryson''
\newline 
\newline
``Computer science education cannot make anybody an expert programmer any
more than studying brushes and pigment can make somebody an expert painter. -
Eric S. Raymond'' 
\newline
\newline
``In theory, theory and practice are the same. In practice,
they’re not. - Yoggi Berra'' 

say something about why.
same something about what you have done in high performance, mixing lab lecture,
also talk about the issues you had in this lab lecture
\begin{itemize}
	\item Write list of needs for a programming class
	\item Present the required learning outcomes of a programming course
	\item Define the structure of this lecture type
	\item Study and predict the possible outcomes of the lecture
	\item Describe how the lecture stimulates deep learning.
	\item Present the tools that the teacher may use in the lecture.
\end{itemize}
% section introduction (end)

\section{Approach} % (fold)
\label{sec:plan}

\url{http://www.socrative.com}\\
\url{https://www.scalable-learning.com/}\\ %this is an in-house website from UU
\url{https://getkahoot.com}\\
\url{https://plickers.com}

% section plan (end)
\section{Analysis} % (fold)
\label{sec:analysis}

% section analysis (end)

\section{Conclusion} % (fold)
\label{sec:conclusion}

% section conclusion (end)
\end{document}
