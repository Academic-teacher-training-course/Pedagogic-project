\documentclass{article}

\usepackage{amsfonts,amsmath,amssymb}
\usepackage{hyperref}

\author{Ali Dorostkar \and Anastasia Kruchinina}
\title{\textsc{A new teaching style in computer programming}}
\date{\today}

\begin{document}

\maketitle \begin{abstract}
In this project we introduce a teaching technique for fields that require
practical work in order to be understood deeply. The focus of this article is 
to provide ways to simulate deep learning among students. To this end we describe 
details of a lecture session tailored specifically for Programming courses
where the students learn by participating, participating and other methods.
\end{abstract}

\section{Introduction} % (fold)
\label{sec:introduction}

Nowdays computers become an essential part of our lives.  ``I have
always wished for my computer to be as easy to use as my telephone; my
wish has come true because I can no longer figure out how to use my
telephone.'' - Bjarne Stroustrup.  

Programming is between the first courses which computer science
students take. There is a variety of programming languages which are
tought in the introduction programming
courses~\cite{de2002language}. Over the past years teacher discussed
ways of teaching programming courses, how to teach and what.

What does it mean to teach programming?

``For a long time it puzzled me how something so expensive, so leading edge,
could be so useless. And then it occurred to me that a computer is a stupid
machine with the ability to do incredibly smart things, while computer
programmers are smart people with the ability to do incredibly stupid things.
They are, in short, a perfect match. - Bill Bryson''

In our opinion, to teach programming means to teach how to interact
with computer, explain your will to it. We should teach students how
to deal with different ``surprises'' which computer may produce, in
particular how to debug the code efficiently.

In~\cite{state_of_art} authors separate a process of writing the
program into two phases. The problem solving phase includes
understanding the problem and designing the algorithm to reach the
solution. This phase requires the knowledge of the computer science
concepts, software design techniques, critical thinking skills and
ability to think abstractly.  In the second phase, called
implementation phase, the ideas obtained in the problem solving phase
are expressed in the executable form, translated into a programming
language.

To master the implementation phase student must learn syntax of the
chosen language and be familiar with development envinronment. It is
common that novice students struggling to learn the most basic
programming constructs, such as conditional and loop statements.



``Computer science education cannot make anybody an expert programmer any
more than studying brushes and pigment can make somebody an expert painter. -
Eric S. Raymond'' 



The tradition approach in teaching programming includes lectures and
laboratory sessions.  During the traditional lecture the teacher
covering the main concepts and showing examples, while students are in
general passive recipients of the information. During the lab sessions
students solve practical assignments and examined individually. It is
common that the large groups of students are involved into the
programming classes. Thus usually students work in small groups (2-3
persons) during the lab sessions. 

According to Wirth~\cite{wirth}, the traditional lecturing has a
number of limitations.  ``Firstly, students’ attention to what the
instructor is saying (i.e. their ability to concentrate) decreases as
the lecture proceeds. This lack of attention manifests itself in a
reduction in the amount of information retained by the student. For
the first 5-10 minutes of a typical 50 minute lecture a student
remembers a high proportion of the information presented, after which
the proportion of information preserved rapidly declines. Students
typically retain 70\% from the first 10 minutes of lecture, and 20\%
from the last 10 minutes.''

Moreover, the amount of the information stored in the long-term memory
depends on the method of presenting informations, in particular in the
amout of interaction with students.  Studies show that just 10\% of
information  is retained in the long-term memory when person is
reading, 50\% when hearing and seeing, 90\% when doing and discussing,
and 95\% when teching and tutoring~\cite{magnesen}.

Therefore, one can conclude that the traditional lecturing does not
suit teaching programming.


``In theory, theory and practice are the same. In practice,
they’re not. - Yoggi Berra'' 




In programming courses students usually have different backgrounds,
different learning habits and different motivation to study. As
pointed out in~\cite{experiment_iceland_2006} ``some of the students
have better preparation than the others and want to start programming
right away and are not willing to follow the teacher’s instructions or
wait for other students. They want to work on their on own pace and
want the teacher to come running to help them when they are stuck.''.
The paper describes an experiment which was performed in the Reykjavik
University. The idea was to remove completely traditional lecture and
instead do more lab sessions (workshops). ``The teachers might give
short presentations, 5 – 10 minutes, at the beginning of the workshop,
or in between, that could be recorded, but mainly the students were
working on their tasks.\ldots When they see they need more knowledge
or help they are can look for recorded material and listen to the
teacher and do not have to wait for him. \ldots Students solve
exercises at very different speed and if they can find help in the
recorded material to keep on working they do not have to wait for the
teacher. The teacher can spend more time on students that need more
help than the recorded lectures can give them or want some extra
explanations; i.e. the teacher has more time for those who ask for
help and really need it.'' The overall experience was positive for
both students and teachers.



===================

say something about why.
same something about what you have done in high performance, mixing lab lecture,
also talk about the issues you had in this lab lecture
\begin{itemize}
	\item Write list of needs for a programming class
	\item Present the required learning outcomes of a programming course
	\item Define the structure of this lecture type
	\item Study and predict the possible outcomes of the lecture
	\item Describe how the lecture stimulates deep learning.
	\item Present the tools that the teacher may use in the lecture.
\end{itemize}
% section introduction (end)

\section{Approach} % (fold)
\label{sec:plan}

\url{http://www.socrative.com}\\
\url{https://www.scalable-learning.com/}\\ %this is an in-house website from UU
\url{https://getkahoot.com}\\
\url{https://plickers.com}

% section plan (end)
\section{Analysis} % (fold)
\label{sec:analysis}




A good overview of the existing programming teaching methods is given
in~\cite{mohorovivcic2011overview}. Authors pointed out the importance
of the deep learning: ``In programming, surface learning can be used
for memorising the language’s syntax, but deep learning is crucial, in
addition to surface learning, for gaining a true understanding of
programming logic and consequently a true competence in
programming''.

The problem-based learning might be a good alternative to the
traditional lecturing.  The problem introduced to the students should
be centered around the real practical problems. Students then discuss
the problem in groups and identify what they need to solve the
problem, what do they know and what should be learned. Then the
teacher present leanguage concepts and elements in the order the
students need to solve the problem. Every time students build
on their work from the week before. This lets students build a
substantial application that does real work.


The lab sessions may be organized in different ways. Students are
provided with a number of tasks to solve. As interesting task could
suggested by a puzzle-based
learning~\cite{mohorovivcic2011overview}. The problem is introduced to
students and then the code is separated into the very small parts
(puzzle pieces). Students must reconstruct the program by putting
parts in the correct order. To procedure can be automated or guided by
a teacher. Another example of the lab task could be to ask students to
provide an example of a program which uses just learned programming
concept.


The teacher may demonstrate to the students a badly program which does
not behave correctly. The goal of such examples is to teach students
the good practices and get the students excited about the development
of skills allowing to write good code.


Biggs presents three levels of thinking about teaching.
Jenkins~\cite{journey_Jenkins} discusses these levels, and pointing
out that every teaching moving through these levels as they gain more
experience. The first level includes trasmitting information, the
responsibility for learning is lying totally in students.  The second
level still includes elements of information transmission, but more
focusing on deeo learning and activities where students actively
participate.  The third, ans the best level, focusing just on
activities engaging students to learn. Teaching is a process of
motivation, the goal is ``to persuade the students that learning to
program (and so programming) be a good thing''. In is pointer out
in~\cite{journey_Jenkins} that in order to learn something, students
must be motivated and they ``must expect to succeed''.






===================









% section analysis (end)

\section{Conclusion} % (fold)
\label{sec:conclusion}


\bibliography{biblio} \bibliographystyle{siam} % achemso, apsrev


% section conclusion (end)
\end{document}
